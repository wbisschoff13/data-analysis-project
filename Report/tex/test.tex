\subsubsection{Introduction}\label{introduction}

A DC motor is to be characterized and a mathematical model is to be
derived.

\subsubsection{First Order Model}\label{first-order-model}

\subsubsection{The Method}\label{the-method}

Calculate the armature resistance $R$ by measuring the armature
current $I$ when supplying voltage $V$ to a short-circuit and
measuring the armature voltage $E$ when supplying voltage to the open
circuit.

Applying an AC voltage and determining the armature voltage and current
allows for the calculation of the armature impedance $Z$. The
inductance $L$ can be calculated from the reactance $X$ of the
impedance.

The back-EMF constant $K_e$ can be determined from

 \begin{equation} 
K_e = \frac{\Delta E}{\Delta\omega}, 
 \end{equation} 
 and the torque constant
$K_t$ can be determined as $K_t = K_e$.

The damping coefficient $b$ can be determined from

 \begin{equation} 
b\omega = K_tI_a, 
 \end{equation} 
 which is derived from 
 \begin{equation} 
T=K_ti
 \end{equation} 
 and

 \begin{equation} 
T = J\dot\omega +b\omega, 
 \end{equation} 
 while keeping the system at a steady
state so that $\dot\omega = 0$.

The motor inertia $J$ can be calculated from

 \begin{equation} 
J = -\frac{T}{\frac{dw}{dt}}.
 \end{equation} 


\subsubsection{Results}\label{results}

The resistance $R = 10,42\Omega$ can be calculated from Table 1.

\begin{longtable}[]{@{}lll@{}}
\toprule
$V_f$ & $E_a$ & $I_a$\tabularnewline
\midrule
\endhead
0 & 19.10 & 1,90\tabularnewline
4 & 23,03 & 2,25\tabularnewline
8 & 26,19 & 2,58\tabularnewline
12 & 29,82 & 2,81\tabularnewline
16 & 34,16 & 3,13\tabularnewline
20 & 37,67 & 3,48\tabularnewline
\bottomrule
\end{longtable}

The back-EMF constant and the torque constants can be calculated as
$K_t = K_e = 0.728$, the damping coefficient can be calculated as
$b = 0.014$. The inductance $L = 0.15$.

\begin{longtable}[]{@{}lll@{}}
\toprule
Ea & Ia & w\tabularnewline
\midrule
\endhead
24 & 1,186 & 12,15\tabularnewline
38 & 1,408 & 28,48\tabularnewline
55 & 1,605 & 49,95\tabularnewline
67 & 1,67 & 66,71\tabularnewline
76 & 1,705 & 79,27\tabularnewline
84 & 1,72 & 89,12\tabularnewline
92 & 1,738 & 100,16\tabularnewline
96 & 1,747 & 109,01\tabularnewline
105 & 1,752 & 120,32\tabularnewline
115 & 1,78 & 135,40\tabularnewline
128 & 1,76 & 155,72\tabularnewline
\bottomrule
\end{longtable}

The inertia $J = 0,022$.

The transfer function $P(s)$ can be determined as


 \begin{equation} 
P(s)=\frac{\omega(s)}{V(s)}=\frac{K_t}{(Js+b)(Ls+R)+K_eK_t}.
 \end{equation} 

